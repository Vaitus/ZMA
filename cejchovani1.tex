%%% Hlavní soubor. Zde se definují základní parametry a odkazuje se na ostatní části. %%%

%% Verze pro jednostranný tisk:
% Okraje: levý 40mm, pravý 25mm, horní a dolní 25mm
% (ale pozor, LaTeX si sám přidává 1in)
\documentclass[12pt,a4paper]{report}
\setlength\textwidth{130mm}
\setlength\textheight{247mm}
\setlength\oddsidemargin{15mm}
\setlength\evensidemargin{15mm}
\setlength\topmargin{0mm}
\setlength\headsep{0mm}
\setlength\headheight{0mm}
% \openright zařídí, aby následující text začínal na pravé straně knihy
\let\openright=\clearpage

%% Pokud tiskneme oboustranně:
% \documentclass[12pt,a4paper,twoside,openright]{report}
% \setlength\textwidth{145mm}
% \setlength\textheight{247mm}
% \setlength\oddsidemargin{15mm}
% \setlength\evensidemargin{0mm}
% \setlength\topmargin{0mm}
% \setlength\headsep{0mm}
% \setlength\headheight{0mm}
% \let\openright=\cleardoublepage

%% Pokud používáte csLaTeX (doporučeno):
%\usepackage{czech}
%% Pokud nikoliv:
\usepackage[czech]{babel}
%\usepackage[T1]{fontenc}
\usepackage{mwe}

%% Použité kódování znaků: obvykle latin2, cp1250 nebo utf8:
\usepackage[utf8]{inputenc}

%% Ostatní balíčky
\usepackage{graphicx}
\usepackage{amsthm}
\usepackage{amsmath}
\usepackage{ mathrsfs }

%% Balíček hyperref, kterým jdou vyrábět klikací odkazy v PDF,
%% ale hlavně ho používáme k uložení metadat do PDF (včetně obsahu).
%% POZOR, nezapomeňte vyplnit jméno práce a autora.
\usepackage[ps2pdf,unicode]{hyperref}   % Musí být za všemi ostatními balíčky
\hypersetup{pdftitle=Název práce}
\hypersetup{pdfauthor=Jméno Příjmení}

%%% Drobné úpravy stylu

% Tato makra přesvědčují mírně ošklivým trikem LaTeX, aby hlavičky kapitol
% sázel příčetněji a nevynechával nad nimi spoustu místa. Směle ignorujte.
\makeatletter
\def\@makechapterhead#1{
  {\parindent \z@ \raggedright \normalfont
   \Huge\bfseries \thechapter. #1
   \par\nobreak
   \vskip 20\p@
}}
\def\@makeschapterhead#1{
  {\parindent \z@ \raggedright \normalfont
   \Huge\bfseries #1
   \par\nobreak
   \vskip 20\p@
}}
\makeatother

% Toto makro definuje kapitolu, která není očíslovaná, ale je uvedena v obsahu.
\def\chapwithtoc#1{
\chapter*{#1}
\addcontentsline{toc}{chapter}{#1}
}

\begin{document}

% Trochu volnější nastavení dělení slov, než je default.
\lefthyphenmin=2
\righthyphenmin=2

%%% Titulní strana práce

\pagestyle{empty}

\begin{center}
\LARGE Základy měření 
\vfill
{\LARGE 2.2.5 Cejchování ručkového voltmetru }

\vfill

\end{center}

13.3.2017 \hfill Vít Teřl

\newpage

\section*{Zadání} 
\begin{enumerate}
    \item Zjistěte deklarovanou třídu přesnosti, rozsahy a měřící systém předloženého 
měřidla. Posuďte jeho celkový mechanický a elektrický stav.
    \item U voltmetru vyberte jeden stejnosměrný rozsah a zkontrolujte jej pomocí 
normálového přístroje. Vyneste korekční křivku.
    \item Ze zjištěných hodnot dopočítejte třídu přesnosti a porovnejte s
hodnotou na stupnici 
přístroje.
\end{enumerate}

\section*{Teoretický úvod}
Pro zjišťování přesnosti a správnosti měření napětí jsou normami předepsaná pravidla. Ověřování funkce a kalibrace voltmetru nazýváme cejchováni. Zjišťují se výchylky například dvou měřících přístrojů při postupném zvyšování měřené veličiny a při následném snižování měřené veličiny. Před začátkem měření je si třeba určit, taký měřící přístroj je normálový a jaký je ověřovaný. Pak pro ověřovaný voltmetr můžeme napsat vztah:
\begin{equation}
    \text{U}_s = \frac{\text{U}_{s+}+\text{U}_{s-}}{2},
\end{equation}
kde  $\text{U}_{S+}[\text{V}]$ - skutečná hodnota, napětí změřené normálem ve směru nahoru,
    $\text{U}_{S-}[\text{V}]$ - skutečná hodnota, napětí změřené normálem ve směru dolů,
    $\text{U}_S[\text{V}]$  - aritmetický průměr hodnot $U_{S+}$ a $U_{S-}$. \\
A korekci vypočteme:
\begin{equation}
    K=U_S-U_N,
\end{equation}
kde  $U_N[\text{V}]$ - nastavovaná hodnota a $K[\text{V}]$ - korekce. \\
Dopočtení třídy přesnosti pro dané měřidlo:
\begin{equation}
    t_p = \frac{|\triangle_{max}|}{M}100,
    \label{tp}
\end{equation}
kde M je rozsah.

\section*{Postup měření}
Daný rozsah se rozdělí na celistvý počet intervalů. Před připojením napájecího 
napětí se zkontroluje nula přístroje.
Zvolené hodnoty se nastavují na ověřovaném přístroji a skutečná napětí pak odečítají 
z
normálového voltmetru. Pro vymezení tření v
ložiskách elektromechanického přístroje se 
měření provádí dvakrát (ve směru zvětšování napětí a ve směru dolů). Při nastavování hodnot 
je třeba dbát, abychom nastavovanou hodnotu nepřesáhli a nevraceli se z
druhé strany. V 
takovém případě je nutné vrátit se o delší úsek stupnice a najíždět na nastavovanou hodnotu 
znova. 
\section*{Měřící přístroje}
\begin{itemize}
    \item Vn - cejchované měřidlo – nutno uvést typ a výrobní/inventární číslo
    \item Vs - normálové měřidlo – nutno uvést typ a třídu přesnosti
    \item P  - potenciometr
    \item SZ - stabilizovaný zdroj
\end{itemize}

\section*{Zpracování výsledků}
\subsection*{Naměřené a dopočítané hodnoty}

\makebox[\textwidth][c]{
\begin{tabular}{|c|c|c|c|c|c|c|c|c|c|c|}
    \hline
    U_N[V] & 1 & 2 & 3 & 4 & 5 & 6 & 7 & 8 & 9 & 10 \\
    \hline
    \alpha[d] & 10 & 20 & 30 & 40 & 50 & 60 & 70 & 80 & 90 & 100\\
    \hline
    U_{S+}[V] & 1,00908 & 1,9883 & 2,976 & 3,964 & 4,9708 & 5,9233 & 6,9563 & 7,9363  & 8,9325 & 9,9635 \\
    \hline
    U_{S-}[V] & 1,04280  &2,0268 &3,0096 &3,9836 &4,9683 &5,9695 &6,9526 &7,9545 &8,9569& 9,964 \\
    \hline
    U_S[V] & 1,02594 &2,00755 &2,9928 &3,9738& 4,96955 &5,9464 &6,95445 &7,9454 &8,9447& 9,96375  \\
    \hline
    K[V] &  0,02594 &0,00755 &-0,0072 &-0,0262 &-0,03045 &-0,0536&-0,04555 & -0,0546 &-0,0553& -0,03625\\
    \hline
\end{tabular}
}
\noindent
    $\alpha[d]$ - výchylka cejchovaného měřidla, v
dílcích stupnice \\

\begin{figure}
    \centering
    \includegraphics[width=0.9\textwidth{}]{korekce.png}
    \caption{Koreční křivka}
    \label{fig:korecke}
\end{figure}

\noindent
Dopočtení třídy přesnosti pro dané měřidlo dosazením do (\ref{tp}) získáme $t_p=0,553$.


\section*{Závěr}
Zkoumali jsme třídu přesnosti ručkového voltmetru, kontrolováním rozdílu od normálového přístroje. Vynesli jsme korekční křivku a vypočetli reálnou třídu přesnosti $t_p=0,553$ s porovnáním s udanou hodnotou na přístroji $t_p=0,2$  zjistíme, že přístroj už zcela nedodržuje tuto třídu přesnosti. 

\end{document}
